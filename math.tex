\documentclass[pdftex,a4paper]{scrartcl}
\usepackage[utf8]{inputenc}
\usepackage[T1]{fontenc}
\usepackage{amssymb}
\usepackage{amstext}
\usepackage{hyperref}
\usepackage{color}

\title{Thoughts on Turtle Graphics with Euler Spirals}
\author{Uli Schlachter}

\newcommand{\rotate}{\text{rotate}}
\newcommand{\total}{\text{total\_rotate}}

\begin{document}

\maketitle

\section{Problem Statement}
TL;DR: \url{https://www.youtube.com/watch?v=Fx1d8x0gIu4} and \url{https://github.com/schneirob/turtlefun}.

We have an angle \(\theta\in\mathbb{R}\). For convenience, the number 1 will represent a whole rotation\footnote{I have
been told that this implies \(2\pi = 1\). Go away with your radians. :-P}. Thus, e.g. \(1/4\) represents 90° in this
representation. We now run a turtle graphics program by infinitely repeating two steps: Move forward by one unit. Rotate
by \(i\cdot\theta\), where \(i\) is the number of steps done so far.

I highly recommend the above YouTube video instead of this short descriptions. It better introduces the problem and has
pretty pictures.

The question is at which point this process becomes periodic. And whether the resulting set of points that were visit is
finite\footnote{In the video, this is referred to as returning home.} or infinite\footnote{This is called a line in the
video.}.

\section{A closer look at the visited Angles}
From the point of view of the turtle, the rotation done in step \(i\) is \(\rotate(i)=i\cdot\theta\). From the point of
view of a spectator, the rotations are cumulative. This defines \(\total(i)\) as:
\[
\total(i)=\sum_{j=1}^i \rotate(j) = \theta\cdot\sum_{j=1}^i j = \theta\cdot i\cdot(i+1)\cdot\frac{1}{2}
\]
We are curious in cases where a certain image is repeated. A certain image is repeated if the same sequence of angles
are repeated.
I.e there must be a \(o\in\mathbb{N}\) with \(o>0\) so that\footnote{This only makes sense if e.g.\ one full rotation is
equivalent to two rotations. Put differently, our angles actually live in \(\mathbb{R}/\mathbb{Z}\). See
\url{https://en.wikipedia.org/wiki/Quotient_group} if you really must. \url{https://en.wikipedia.org/wiki/Circle_group}
also seems loosely related. Surely the circle group (with multiplcaition) and \(\mathbb{R}/\mathbb{Z}\) (with addition)
are isomorphic to each other with \(x\mapsto e^{2\cdot\pi x}\).} \(\rotate(i)=\rotate(i+o)\) for all \(i\in\mathbb{N}\). Thus, we also have
\(0=\rotate(0)=\rotate(o)\) and so we are looking for cases where \(\rotate(o)\) is no rotation at all.

\section{Irrational Numbers}
For an irrational \(\theta\), \(\rotate\) cannot become periodic and the turtle will never return home.

I suppose: The resulting images will be quite chaotic and even harder to draw without falling prey to bad rounding
errors.

\section{Rational Numbers (Old and Wrong)}
\(\theta\in\mathbb{Q}\) means that there are coprime \(p,q\in\mathbb{Z}\) so that \(\theta=\frac{p}{q}\). Then,
\(\rotate(q)=\theta\cdot q=p\) is the \textcolor{red}{first}\footnote{This is where this section is wrong. See the next
section for the correct(?) computation. The rest of this section still seems to be true.} multiple of \(\theta\)
representing a whole rotation. What is the total
rotation \(\total(q)\) at this point? Insertion yields \(\total(q)=\frac{p}{q}\cdot q\cdot(q+1)\cdot \frac{1}{2}
=p\cdot(q+1)\cdot\frac{1}{2}\).
Let's do a case analysis.

\subsection{$p$ and $q$ are both even}
This case cannot happen since we assumed \(p\) and \(q\) to be coprime.

\subsection{$q$ is odd and $p$ is either even or odd}
In this case \(q+1\) is even so \(\total(q)=p\cdot\frac{q+1}{2}\) is a whole number.

Thus, when the sequence of angles is repeated, the turtle will have finished a whole number of rotations. If, and only
if, the turtle returned to its home at this point, will the set of visited points be finite.

Conjecture: These are the lines. It might be possible that the process ends up home in step \(q\), but that seems
unlikely and would produce an image that looks different than the usual images. I conjecture this does not actually
occur.

\subsection{$q$ is even and $p$ is odd}
In this case \(\total(q)\) will be a whole number plus \(1/2\), because the product of two odd numbers is odd.

Thus, when the sequence of angles is repeated, the turtle will have done have a rotation. The next period will then draw
the same image again, but rotated by 180°. Afterwards, it will really repeat the same image again. Put differently,
after exactly \(2q\) steps does the image become repeated.

Conjecture: In the YouTube video, all the finite cases look like this with a spiral after half of this. Thus, I should
look at step \(q/2\). I suppose that \(\rotate(\frac{q}{2} + i)=\rotate(\frac{q}{2} -i)\), which would be why the turtle
traces its steps backwards in this case. At least \(\rotate(\frac{q}{2})=\theta\cdot \frac{q}{2} = p/q \cdot \frac{q}{2}
= p/2\), so the turtle does a 180° rotation at the middle. Also, \(\rotate(\frac{q}{2} + i)+\rotate(\frac{q}{2}-i) =
\theta(\frac{q}{2}+i + \frac{q}{2}-i) = \theta(q) = 2\rotate(\frac{q}{2})\).  Since \(\rotate(\frac{q}{2})\) is half a
rotation, this means the two angles sum to a full circle. Does this already prove that the turtle follows its steps
backwards?

\section{Rational Numbers (New and less developed)}
\(\theta\in\mathbb{Q}\) means that there are numbers \(p,q\in\mathbb{Z}\) so that \(\theta=\frac{p}{q}\).
For \(\rotate(i)=\frac{p}{q}\cdot i\) to be a whole number, \(p\cdot i\) must be a multiple of \(q\). So,
\(i=\frac{q}{\text{gcd}(p,q)}\) is the first time this happens. We thus have:
\[
\rotate(i)=\frac{p}{q}\cdot\frac{q}{\text{gcd}(p,q)}=\frac{p}{\text{gcd}(p,q)}
\]
Let \(a,b\in\mathbb{Z}\) be coprime numbers so that \(\frac{a}{b}=\frac{p}{\text{gcd}(p,q)}\).

From here on, one can continue like in the section above. The only difference is that this now happens at an earlier step
of the computation.

\section{Acknowledgements}
Thanks a lot to Schneirob for producing \url{https://www.youtube.com/watch?v=Fx1d8x0gIu4}. This made it really easy for
me to get an overview over things and identify some interesting angels. Then I used
\url{https://github.com/schneirob/turtlefun} to get the angles that occur during the Euler spiral. All the hard parts
were already prepared for me.

\end{document}
