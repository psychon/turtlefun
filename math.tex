\documentclass[pdftex,a4paper]{scrartcl}
\usepackage[utf8]{inputenc}
\usepackage[T1]{fontenc}
\usepackage{amssymb}
\usepackage{amstext}
\usepackage{amsmath}
\usepackage{hyperref}

\title{Thoughts on Turtle Graphics with Euler Spirals}
\author{Uli Schlachter}

\DeclareMathOperator{\rotate}{rotate}
\DeclareMathOperator{\total}{total\_rotate}

\begin{document}

\maketitle

\section{Problem Statement}
TL;DR: \url{https://www.youtube.com/watch?v=Fx1d8x0gIu4} and \url{https://github.com/schneirob/turtlefun}.

We have an angle \(\theta\in\mathbb{R}\). For convenience, the number 1 will represent a whole rotation\footnote{I have
been told that this implies \(2\pi = 1\). Go away with your radians. :-P}. Thus, e.g. \(1/4\) represents 90° in this
representation. We now run a turtle graphics program by infinitely repeating two steps: Move forward by one unit. Rotate
by \(i\cdot\theta\), where \(i\) is the number of steps done so far.

I highly recommend the above YouTube video instead of this short descriptions. It better introduces the problem and has
pretty pictures.

The question is at which point this process becomes periodic. And whether the resulting set of points that were visit is
finite\footnote{In the video, this is referred to as returning home.} or infinite\footnote{This is called a line in the
video.}.

\section{A closer look at the visited Angles}
From the point of view of the turtle, the rotation done in step \(i\) is \(\rotate(i)=i\cdot\theta\). From the point of
view of a spectator, the rotations are cumulative. This defines \(\total(i)\) as:
\[
\total(i)=\sum_{j=1}^i \rotate(j) = \theta\cdot\sum_{j=1}^i j = \theta\cdot i\cdot(i+1)\cdot\frac{1}{2}
\]
We are curious in cases where a certain image is repeated. A certain image is repeated if the same sequence of angles
are repeated.
I.e there must be a \(o\in\mathbb{N}\) with \(o>0\) so that\footnote{This only makes sense if e.g.\ one full rotation is
equivalent to two rotations. Put differently, our angles actually live in \(\mathbb{R}/\mathbb{Z}\). See
\url{https://en.wikipedia.org/wiki/Quotient_group} if you really must. \url{https://en.wikipedia.org/wiki/Circle_group}
also seems loosely related. Surely the circle group (with multiplcaition) and \(\mathbb{R}/\mathbb{Z}\) (with addition)
are isomorphic to each other with \(x\mapsto e^{2\cdot\pi x}\).} \(\rotate(i)=\rotate(i+o)\) for all \(i\in\mathbb{N}\). Thus, we also have
\(0=\rotate(0)=\rotate(o)\) and so we are looking for cases where \(\rotate(o)\) is no rotation at all.

\section{Irrational Numbers}
For an irrational \(\theta\), \(\rotate\) cannot become periodic and the turtle will never return home.

I suppose: The resulting images will be quite chaotic and even harder to draw without falling prey to bad rounding
errors.

\section{Rational Numbers}
\(\theta\in\mathbb{Q}\) means that there are coprime \(p,q\in\mathbb{Z}\) so that \(\theta=\frac{p}{q}\). Then,
\(\rotate(q)=\theta\cdot q=p\) is the first multiple of \(\theta\) representing a whole rotation. What is the total
rotation \(\total(q)\) at this point? Insertion yields \(\total(q)=\frac{p}{q}\cdot q\cdot(q+1)\cdot \frac{1}{2}
=p\cdot(q+1)\cdot\frac{1}{2}\).
Let's do a case analysis.

\subsection{$q$ is odd}
In this case \(q+1\) is even so \(\total(q)=p\cdot\frac{q+1}{2}\) is a whole number.

Thus, when the sequence of angles is repeated, the turtle will have finished a whole number of rotations. If, and only
if, the turtle returned to its home at this point, will the set of visited points be finite.

Conjecture: These are the lines. It might be possible that the process ends up home in step \(q\), but that seems
unlikely and would produce an image that looks different than the usual images. I conjecture this does not actually
occur.

\subsection{$q$ is even}
If \(q\) is even, then \(p\) must be odd, because we are assuming these two numbers to be coprime.
In this case\footnote{More detailed: Let \(a,b\in\mathbb{Z}\) be numbers so that \(q=2a\) and \(p=2b+1\). We have
\(\total(q)=p\cdot(q+1)\cdot\frac{1}{2}=(2b+1)\cdot(2a+1)\cdot\frac{1}{2}=2ab+a+b+\frac{1}{2}\). Everything but 1/2 is
in \(\mathbb{Z}\).}
\(\total(q)\) will be a whole number plus \(1/2\), because the product of two odd numbers is odd.

Thus, when the sequence of angles is repeated, the turtle will have turned in the opposite direction that it started
from. The next period will then draw
the same image again, but rotated by 180°. Afterwards, it will really repeat the same image again. Put differently,
after exactly \(2q\) steps does the image become repeated.

Let us look at what happens around step \(q/2\). We have \(\rotate(\frac{q}{2})=\theta\cdot \frac{q}{2}
= \frac{p}{q} \cdot \frac{q}{2} = p/2\). Since \(p\) is a whole number, this must be half a rotation, i.e.\ 180° and the
turtle turns around.

What does the turtle do after turning around? Let us look at step \(\frac{q}{2}+i\) and do some mathmagical rearranging
on
\(
\rotate\left(\frac{q}{2}+i\right)
=\theta\left(\frac{q}{2}+i\right)
\):
\[
\rotate\left(\frac{q}{2}+i\right)
=\theta\left(q-\left(\frac{q}{2}-i\right)\right)
=2\theta\cdot \frac{q}{2} - \theta\cdot\left(\frac{q}{2}-i\right)
=2\rotate\left(\frac{q}{2}\right)-\rotate\left(\frac{q}{2}-i\right)
\]
Since \(\rotate(\frac{q}{2})\) is half a rotation, \(2\cdot\rotate(\frac{q}{2})\) must be a whole rotation, i.e.\ 0°. Thus,
we arrive at \(\rotate(\frac{q}{2}+i)=-\rotate(\frac{q}{2}-i)\).

What does this mean? It means that after turning around, the turtle will follow its steps backwards home! In step
q, it will arrive back at the place where it started, but turned by 180°. Then, it traces the same path in the opposite
direction and at step \(2\cdot q\) it will arrive back at its starting position with its starting angle.

\section{Taking the Position into Account}
Let us represent the position of the turtle in step \(i\) with \(p_i\in\mathbb{C}\) and its direction with
\(d_i\in\mathbb{C}\) with \(|d_i|=1\), i.e.\ \(d_i\) is a point on the unit circle. Then, at step \(i+1\), the turtle
will do:
\begin{align*}
p_{i+1} &= p_i + d_i &
d_{i+1} &= d_i \cdot e^{i\cdot \theta\cdot 2\pi}
\end{align*}
Defining \(p_0=0\) and \(d_0=1=e^0\), we can arrive at:
\[
d_i = e^{\theta\cdot 2\pi\cdot \sum_{j=0}^i j}
= e^{\theta\cdot i\cdot(i+1)\frac{1}{2}\cdot 2\pi}
\]
This means for the position:
\[
p_i = \sum_{j=0}^i d_j
= \sum_{j=0}^i e^{\theta\cdot j\cdot(j+1)\frac{1}{2}\cdot 2\pi}
= e^{\theta\cdot \pi\cdot\prod_{j=0}^i j\cdot (j+1) }
= e^{\theta\cdot \pi\cdot 2\cdot i! \cdot (i+1)! }
= e^{\theta\cdot \pi\cdot 2\cdot (i!)^2 \cdot (i+1) }
\]
We arrive at an expression containing a squared factorial in the exponent. And if we want to know when the turtle
returns home, we would have to solve this whole expression for being equal to zero. I have no good ideas on how to
continue here.

One good idea: Test whether this formula actually holds. I have only derived it, but never tested it.

\section{Acknowledgements}
Thanks a lot to Schneirob for producing \url{https://www.youtube.com/watch?v=Fx1d8x0gIu4}. This made it really easy for
me to get an overview over things and identify some interesting angels. Then I used
\url{https://github.com/schneirob/turtlefun} to get the angles that occur during the Euler spiral. All the hard parts
were already prepared for me.

\end{document}
